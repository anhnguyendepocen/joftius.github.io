\documentclass[11pt]{article}
\usepackage{fullpage}
\usepackage[left=1in,top=1in,right=1in,bottom=1in,headheight=3ex,headsep=3ex]{geometry}
\usepackage{graphicx}
\usepackage{float}

\newcommand{\blankline}{\quad\pagebreak[2]}
%%%%%%%%%%%%%%%%%%%%%%%%%%%%%%%%%%%%%%%%%%%%%%%%%%%%%%%%%%%%%%

% Modify Course title, instructor name, semester here %%%%%%%%

\title{Syllabus: Stats UB-103}
\author{Professor Joshua Loftus}
\date{Fall 2019}

%%%%%%%%%%%%%%%%%%%%%%%%%%%%%%%%%%%%%%%%%%%%%%%%%%%%%%%%%%%%%%

% Don't touch this %%%%%%%%%%%%%%%%%%%%%%%%%%%%%%%%%%%%%%%%%%%
\usepackage[sc]{mathpazo}
\linespread{1.05} % Palatino needs more leading (space between lines)
\usepackage[T1]{fontenc}
\usepackage[mmddyyyy]{datetime}% http://ctan.org/pkg/datetime
\usepackage{advdate}% http://ctan.org/pkg/advdate
\newdateformat{syldate}{\twodigit{\THEMONTH}/\twodigit{\THEDAY}}
\newsavebox{\MONDAY}\savebox{\MONDAY}{Mon}% Mon
\newcommand{\week}[1]{%
%  \cleardate{mydate}% Clear date
% \newdate{mydate}{\the\day}{\the\month}{\the\year}% Store date
  \paragraph*{\kern-2ex\quad #1, \syldate{\today} - \AdvanceDate[4]\syldate{\today}:}% Set heading  \quad #1
%  \setbox1=\hbox{\shortdayofweekname{\getdateday{mydate}}{\getdatemonth{mydate}}{\getdateyear{mydate}}}%
  \ifdim\wd1=\wd\MONDAY
    \AdvanceDate[7]
  \else
    \AdvanceDate[7]
  \fi%
}
\usepackage{setspace}
\usepackage{multicol}
%\usepackage{indentfirst}
\usepackage{fancyhdr,lastpage}
\usepackage{url}
\pagestyle{fancy}
\usepackage{hyperref}
\usepackage{lastpage}
\usepackage{amsmath}
\usepackage{layout}

\lhead{}
\chead{}
%%%%%%%%%%%%%%%%%%%%%%%%%%%%%%%%%%%%%%%%%%%%%%%%%%%%%%%%%%%%%%

% Modify header here %%%%%%%%%%%%%%%%%%%%%%%%%%%%%%%%%%%%%%%%%
\rhead{}

%%%%%%%%%%%%%%%%%%%%%%%%%%%%%%%%%%%%%%%%%%%%%%%%%%%%%%%%%%%%%%
% Don't touch this %%%%%%%%%%%%%%%%%%%%%%%%%%%%%%%%%%%%%%%%%%%
\lfoot{}
%\cfoot{\small \thepage/\pageref*{LastPage}}
\rfoot{}

\usepackage{array, xcolor}
\usepackage{color,hyperref}
\definecolor{clemsonorange}{HTML}{EA6A20}
\hypersetup{colorlinks,breaklinks,linkcolor=clemsonorange,urlcolor=clemsonorange,anchorcolor=clemsonorange,citecolor=black}

\begin{document}

\maketitle

\blankline

\begin{tabular*}{.93\textwidth}{@{\extracolsep{\fill}}lr}

%%%%%%%%%%%%%%%%%%%%%%%%%%%%%%%%%%%%%%%%%%%%%%%%%%%%%%%%%%%%%%

  % Modify information %%%%%%%%%%%%
  %%%%%%%%%%%%%%%%%%%%%%%%%%%%% 
E-mail: \texttt{loftus@nyu.edu} & Web: \url{http://joshualoftus.com/page/introstats/}  \\

 Office hours: email for appointment  &  Class hours: Tu/Th/F 3:30-4:45pm \\

 Office: KMC 8-60 & Classroom: Tisch 201 \\
TF: Vinu Abeywickrama \\
TF E-mail: \texttt{va774@stern.nyu.edu} & \\
TF Office hours: TBD &  \\
TF Office: KMC 8-174 \\
\hline
\end{tabular*}

\vspace{5 mm}

% First Section %%%%%%%%%%%%%%%%%%%%%%%%%%%%%%%%%%%%%%%%%%%%

\section*{Course Description}

This course examines modern statistical methods as a basis for decision making in the face of uncertainty. Topics include probability theory, discrete and continuous distributions, hypothesis testing, estimation, and statistical quality control. With the aid of computers, these statistical methods are used to analyze data. Also presented are an introduction to statistical models and their application to decision making. Topics include the simple linear regression model, inference in regression analysis, sensitivity analysis, and multiple regression analysis.

\subsection*{Exam schedule}

Tentative dates for in-class midterms: TBD.

\bigskip

% Second Section %%%%%%%%%%%%%%%%%%%%%%%%%%%%%%%%%%%%%%%%%%%

\section*{Required Materials}

\begin{itemize}
\item We will make use of a number of free resources linked on the course page.
\item On certain days students will be asked to bring a computer to class. 
\end{itemize}

% Third Section %%%%%%%%%%%%%%%%%%%%%%%%%%%%%%%%%%%%%%%%%%%

% Fourth Section %%%%%%%%%%%%%%%%%%%%%%%%%%%%%%%%%%%%%%%%%%%

\section*{Tentative Outline}

Items listed under ``topics'' and the last section, on machine learning, should be considered outside of the core concepts of the class. They are included to indicate the context and importance of the core concepts in the ``modern'' era of data science and machine learning. Course assessments will not require any advanced knowledge of these topics.

\begin{enumerate}
\item Measurement and experiments

  \begin{itemize}
  \item Reliability and validity
  \item Randomized controlled trials
  \item Observational studies, confounding
  \end{itemize}
  
\item Description and visualization

  \begin{itemize}
  \item Summarizing data
  \item Histograms
  \item Other visualizations
  \end{itemize}
  
\item Probability basics

  \begin{itemize}
  \item History, games, sets, counting
  \item Random variables, expectation, variance
  \item Theorems: Chebyshev, Bayes
  \end{itemize}
  
\item Probability in statistics

  \begin{itemize}
  \item Estimation, bias and variance
  \item Sampling distributions, central limit theorem
  \item Topics: Stein's paradox, error propagation
  \end{itemize}
  
\item Confidence intervals and tests

  \begin{itemize}
  \item Intervals, error bars, $t$-distribution
  \item Interpretation
  \item Hypotheses, falsification
  \item Bootstrap and resampling methods
  \item Topics: multiple testing
  \end{itemize}
  
\item Regression

  \begin{itemize}
  \item Covariance, correlation, simple regression
  \item Prediction, intervals, tests, and interpretation
  \item Non-linearity, transformations, discrete predictors
  \item Multiple regression
  \item Topics: logistic regression, mixed effects models, causal inference
  \end{itemize}

\item Machine learning

  \begin{itemize}
  \item Model selection, complexity, validation
  \item Classification
  \end{itemize}
  
\end{enumerate}

The order of this outline is not intended to be chronological. We may revisit certain topics at various points throughout the course.

% Fifth Section %%%%%%%%%%%%%%%%%%%%%%%%%%%%%%%%%%%%%%%%%%%

\section*{Course Policies}

\subsection*{General Rules}

\begin{itemize}
\item Be on time to class. Attendance is mandatory and part of the grade.
\item During lectures, everyone is encouraged to \underline{ask and answer questions} and actively participate, but please raise your hand first. After you have already spoken once try to be mindful of giving other students time to speak as well.
\item During lectures, electronic devices may be used \underline{only} for taking notes unless otherwise specified. If you are distracting your classmates or not paying attention I will stop the lecture and ask you to put the device away.
\item Late submission of assignments may be permitted with prior notice on a case-by-case basis, but is generally \underline{highly discouraged}. There is no guarantee of acceptance of late submissions even for partial credit.
\item Students are encouraged to form study groups review course material together, but unless otherwise specified every assignment must be completed alone. Receiving help or copying answers from other students, the internet, or any other source may be considered plagiarism, a very serious violation that may be investigated by the university.
\end{itemize}

These rules and the Code of Conduct mentioned below exist for your benefit and that of your classmates by helping create a positive and constructive environment which is a safe space for everyone to learn.

\subsection*{Grading Policy}

Stern policies regarding grades for undergraduate core courses apply to this class.

To avoid grades which are heavily biased toward specific skills like test-taking, we will use a variety of assessments including in-class tests, problem sets (homework), in-class participation, online participation, and data analysis projects. Before determining final grade weights for these different assessments we will have a survey so everyone can have some input into the decision. Attendance is included under in-class participation.

%% Final grades will be determined as follows:
%% \begin{itemize}
%%     \item \underline{\textbf{30\%}} distributed equally for 2 in class midterms. 
%%     \item \underline{\textbf{30\%}} for homework.
%%     \item \underline{\textbf{20\%}} for a final project.
%%     \item \underline{\textbf{20\%}} for participation.
%% \end{itemize}



\subsection*{Code of Conduct}

All students are expected to follow the Stern Code of Conduct \url{http://www.stern.nyu.edu/uc/codeofconduct}. A student's responsibilities include, but are not limited to, the following:
\begin{itemize}
\item A duty to acknowledge the work and efforts of others when submitting work as one's own.
Ideas, data, direct quotations, paraphrasing, creative expression, or any other incorporation
of the work of others must be clearly referenced.
\item A duty to exercise the utmost integrity when preparing for and completing examinations, including an obligation to report any observed violations.
\item To minimize the temptation for copying or sharing during an exam, there will be multiple versions of every exam, and the seating order will be randomly assigned.
\end{itemize}

\subsection*{Accommodations for Different Abilities}
If you are qualified for academic accommodation of any kind during this course, you must notify me at the beginning of the course and provide a letter from the Moses Center for Students with Disabilities (CSD, 998-4980, www.nyu.edu/csd) verifying your registration and outlining the accommodations they recommend.  If you will need to take an exam at the CSD, you must submit a completed Exam Accommodations Form to them at least one week prior to the scheduled exam time to be guaranteed accommodation.

\subsection*{Health and Wellness}
To access the University's health and mental health resources, contact the NYU Wellness Exchange.
You can call its private hotline (212-443-9999), available 24 hours a day, seven days a week, to reach out to a
professional who can help to address day-to-day challenges as well as other health-related concerns.

%\footnotesize{Discrimination based on race, color, religion, creed, sex, national origin, age, disability, veteran status, or sexual orientation is a violation of state and federal law and/or NC State University policy and will not be tolerated. Harassment of any person (either in the form of quid pro quo or creation of a hostile environment) based on race, color, religion, creed, sex, national origin, age, disability, veteran status, or sexual orientation also is a violation of state and federal law and/or NC State University policy and will not be tolerated. Retaliation against any person who complains about discrimination is also prohibited. NC State's policies and regulations covering discrimination, harassment, and retaliation may be accessed at \href{http://policies.ncsu.edu/policy/pol-04-25-05} or  \href{http://www.ncsu.edu/equal_op/}. Any person who feels that he or she has been the subject of prohibited discrimination, harassment, or retaliation should contact the Office for Equal Opportunity (OEO) at 919-515-3148.}

% Course Schedule %%%%%%%%%%%%%%%%%%%%%%%%%%%%%%%%%%%%%%%%%%%


\end{document}
