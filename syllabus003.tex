\documentclass[11pt]{article}
\usepackage{fullpage}
\usepackage[left=1in,top=1in,right=1in,bottom=1in,headheight=3ex,headsep=3ex]{geometry}
\usepackage{graphicx}
\usepackage{float}

\newcommand{\blankline}{\quad\pagebreak[2]}
%%%%%%%%%%%%%%%%%%%%%%%%%%%%%%%%%%%%%%%%%%%%%%%%%%%%%%%%%%%%%%

% Modify Course title, instructor name, semester here %%%%%%%%

\title{Syllabus: Stats UB-003}
\author{Instructor: Joshua Loftus}
\date{Spring 2018}

%%%%%%%%%%%%%%%%%%%%%%%%%%%%%%%%%%%%%%%%%%%%%%%%%%%%%%%%%%%%%%

% Don't touch this %%%%%%%%%%%%%%%%%%%%%%%%%%%%%%%%%%%%%%%%%%%
\usepackage[sc]{mathpazo}
\linespread{1.05} % Palatino needs more leading (space between lines)
\usepackage[T1]{fontenc}
\usepackage[mmddyyyy]{datetime}% http://ctan.org/pkg/datetime
\usepackage{advdate}% http://ctan.org/pkg/advdate
\newdateformat{syldate}{\twodigit{\THEMONTH}/\twodigit{\THEDAY}}
\newsavebox{\MONDAY}\savebox{\MONDAY}{Mon}% Mon
\newcommand{\week}[1]{%
%  \cleardate{mydate}% Clear date
% \newdate{mydate}{\the\day}{\the\month}{\the\year}% Store date
  \paragraph*{\kern-2ex\quad #1, \syldate{\today} - \AdvanceDate[4]\syldate{\today}:}% Set heading  \quad #1
%  \setbox1=\hbox{\shortdayofweekname{\getdateday{mydate}}{\getdatemonth{mydate}}{\getdateyear{mydate}}}%
  \ifdim\wd1=\wd\MONDAY
    \AdvanceDate[7]
  \else
    \AdvanceDate[7]
  \fi%
}
\usepackage{setspace}
\usepackage{multicol}
%\usepackage{indentfirst}
\usepackage{fancyhdr,lastpage}
\usepackage{url}
\pagestyle{fancy}
\usepackage{hyperref}
\usepackage{lastpage}
\usepackage{amsmath}
\usepackage{layout}

\lhead{}
\chead{}
%%%%%%%%%%%%%%%%%%%%%%%%%%%%%%%%%%%%%%%%%%%%%%%%%%%%%%%%%%%%%%

% Modify header here %%%%%%%%%%%%%%%%%%%%%%%%%%%%%%%%%%%%%%%%%
\rhead{}

%%%%%%%%%%%%%%%%%%%%%%%%%%%%%%%%%%%%%%%%%%%%%%%%%%%%%%%%%%%%%%
% Don't touch this %%%%%%%%%%%%%%%%%%%%%%%%%%%%%%%%%%%%%%%%%%%
\lfoot{}
%\cfoot{\small \thepage/\pageref*{LastPage}}
\rfoot{}

\usepackage{array, xcolor}
\usepackage{color,hyperref}
\definecolor{clemsonorange}{HTML}{EA6A20}
\hypersetup{colorlinks,breaklinks,linkcolor=clemsonorange,urlcolor=clemsonorange,anchorcolor=clemsonorange,citecolor=black}

\begin{document}

\maketitle

\blankline

\begin{tabular*}{.93\textwidth}{@{\extracolsep{\fill}}lr}

%%%%%%%%%%%%%%%%%%%%%%%%%%%%%%%%%%%%%%%%%%%%%%%%%%%%%%%%%%%%%%

  % Modify information %%%%%%%%%%%%
  %%%%%%%%%%%%%%%%%%%%%%%%%%%%% 
E-mail: \texttt{loftus@nyu.edu} & Web: \url{http://joshualoftus.com/page/introregression/}  \\

 Office hours: by appointment  &  Class hours: M/W 3:30-4:45pm \\

 Office: KMC 8-66 & Classroom: Tisch UC04 \\
 & \\
&  \\
\hline
\end{tabular*}

\vspace{5 mm}

% First Section %%%%%%%%%%%%%%%%%%%%%%%%%%%%%%%%%%%%%%%%%%%%

\section*{Course Description}

The objective of this course is to introduce students to the basic statistical techniques that are widely used in business and other fields. In particular, considerable attention will be devoted to the technique of regression analysis, which is a useful and powerful technique for modeling the relationships between variables of interest.

\subsection*{Exam schedule}

The final exam is scheduled for May 9, from 4-5:50pm. 
The in-class midterm exam will be held on April 18.

\bigskip

% Second Section %%%%%%%%%%%%%%%%%%%%%%%%%%%%%%%%%%%%%%%%%%%

\section*{Required Materials}

\begin{itemize}
\item We will make use of a number of free resources linked on the course page.
\item On certain days students will be asked to bring a laptop to class. 
\end{itemize}

% Third Section %%%%%%%%%%%%%%%%%%%%%%%%%%%%%%%%%%%%%%%%%%%

% Fourth Section %%%%%%%%%%%%%%%%%%%%%%%%%%%%%%%%%%%%%%%%%%%

\section*{Tentative Outline}

Items listed under ``topics'' and the last section, on machine learning, should be considered outside of the core concepts of the class. They are included to indicate the context and importance of the core concepts in the ``modern'' era of data science and machine learning. Homework and exam questions will not require specific knowledge of these topics, and will only relate to the topics at most by asking how a particular core concept can be applied to them.

\begin{enumerate}
  
\item Statistics review

  \begin{itemize}
  \item Notation and random variables
  \item Description and visualization
  \item Sampling and central limit theorem
  \item Intervals, tests, bootstrap
  \item Topics: multiple testing
  \end{itemize}
  
\item Regression

  \begin{itemize}
  \item Covariance, correlation, simple regression
  \item Prediction, intervals, tests, and interpretation
  \item Non-linearity, transformations, discrete predictors
  \item Multiple regression
  \item Topics: logistic regression, mixed effects models, causal inference
  \end{itemize}

\item Machine learning

  \begin{itemize}
  \item Model selection, complexity, validation
  \item Classification
  \item Unsupervised learning
  \end{itemize}
  
\end{enumerate}

% Fifth Section %%%%%%%%%%%%%%%%%%%%%%%%%%%%%%%%%%%%%%%%%%%

\section*{Course Policies}

\subsection*{General Rules}

\begin{itemize}
\item Be on time to class. Attendance is part of the grade.
\item Do not use mobile devices in class. Laptops may be used only for taking notes.
\item Late submission of assignments may be permitted with prior notice and on a case-by-case basis, but is generally \underline{highly discouraged} and may result in half credit for the assignment.
\end{itemize}

\subsection*{Grading Policy}

Final grades will be determined as follows.
\begin{itemize}
    \item \underline{\textbf{20\%}} for one in class midterm. 
    \item \underline{\textbf{30\%}} for a comprehensive final exam.
    \item \underline{\textbf{20\%}} for homework.
    \item \underline{\textbf{20\%}} for a project.
    \item \underline{\textbf{10\%}} for participation.
\end{itemize}

Participation will be determined mostly by objective measures like attendance.

\subsection*{Code of Conduct}

All students are expected to follow the Stern Code of Conduct \url{http://www.stern.nyu.edu/uc/codeofconduct}. A student's responsibilities include, but are not limited to, the following:
\begin{itemize}
\item A duty to acknowledge the work and efforts of others when submitting work as one's own.
Ideas, data, direct quotations, paraphrasing, creative expression, or any other incorporation
of the work of others must be clearly referenced.
\item A duty to exercise the utmost integrity when preparing for and completing examinations, including an obligation to report any observed violations.
\item To minimize the temptation for copying or sharing during an exam, there will be multiple versions of every exam, and the seating order will be randomly assigned.
\end{itemize}

\subsection*{Accommodations for Different Abilities}
If you are qualified for academic accommodation of any kind during this course, you must notify me at the beginning of the course and provide a letter from the Moses Center for Students with Disabilities (CSD, 998-4980, www.nyu.edu/csd) verifying your registration and outlining the accommodations they recommend.  If you will need to take an exam at the CSD, you must submit a completed Exam Accommodations Form to them at least one week prior to the scheduled exam time to be guaranteed accommodation.

\subsection*{Health and Wellness}
To access the University's health and mental health resources, contact the NYU Wellness Exchange.
You can call its private hotline (212-443-9999), available 24 hours a day, seven days a week, to reach out to a
professional who can help to address day-to-day challenges as well as other health-related concerns.

% Course Schedule %%%%%%%%%%%%%%%%%%%%%%%%%%%%%%%%%%%%%%%%%%%


\end{document}
